%!TEX TS-program = xelatex
%!TEX encoding = UTF-8 Unicode

\documentclass[12pt]{article}
\usepackage{geometry}
\geometry{letterpaper}
\usepackage{graphicx}
\usepackage{amssymb}
\usepackage{minted}

\usepackage{fontspec,xltxtra,xunicode}
\defaultfontfeatures{Mapping=tex-text}
\setromanfont[Mapping=tex-text]{Didot}
\setsansfont[Scale=MatchLowercase,Mapping=tex-text]{Gill Sans}
\setmonofont[Scale=MatchLowercase]{Panic Sans}

\definecolor{codebg}{rgb}{0.95,0.95,0.95}

\newminted{tranquil}{linenos=true, mathescape, tabsize=1}

\title{Tranquil Programming Language}
\author{Fjölnir Ásgeirsson}
\date{19th of June, 2012.} 

\begin{document}
\maketitle

\section{Syntax}

\subsection{Comments}
\paragraph{}
Comments start at a backslash and end before a new line.
\begin{tranquilcode}
	\ A comment.
\end{tranquilcode}

\subsection{Variables}
\paragraph{}
Variables are defined using a simple assignment and are always local to the current block scope.

\begin{tranquilcode}
	variable = 123
\end{tranquilcode}

\subsection{Blocks}
\paragraph{}
Blocks are an important part of Tranquil. They define a ``block'' of code in the current scope, that can be executed at a later time.

\begin{tranquilcode}
	\ A simple block
	aBlock = {
		\ Block body
	}
	\ A block that takes two arguments
	anotherBlock = { argumentOne, argumentTwo |
		\ Block body
	}
\end{tranquilcode}

\subsubsection{Calling Blocks}
\paragraph{}
In order to execute a block one must call it, passing any arguments it may accept.

\begin{tranquilcode*}{firstnumber=9}
	anotherBlock(1, 2)
\end{tranquilcode*}

\subsection{Objects}
\paragraph{}
In Tranquil there are no types. Every value is an object. Objects instances of a class, who themselves are object instances of 'Class'.

\subsubsection{Classes}
\begin{tranquilcode}
	class MyClass
	end
\end{tranquilcode}

\subsubsection{Class Methods}
\begin{tranquilcode}
	class MyClass
		+ aClassMethod: argument {
		}
	end
\end{tranquilcode}

\subsubsection{Instance Methods}
\begin{tranquilcode}
	class MyClass
		- anInstanceMethod: argument {
		}
	end
\end{tranquilcode}

\subsubsection{Object Properties}
Objects have ``Properties'', that is pieces of data associated with a key on the object.

\begin{tranquilcode}
	anInstance#property = 123
	variable = anInstance#property
\end{tranquilcode}

\subsection{Messages}
\paragraph{}
To make an object perform an action (Execute a method) you must send it a message that triggers said action.

\begin{tranquilcode}
	anInstance = aClass new.
	anInstance methodWithAnArgument: argument.
\end{tranquilcode}




\end{document}  